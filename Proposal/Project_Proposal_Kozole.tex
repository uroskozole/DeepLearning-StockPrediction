\documentclass{article}

% % Packages
% \usepackage[utf8]{inputenc} % UTF-8 encoding
% \usepackage{amsmath} % Math formatting
% \usepackage{graphicx} % For including images
% \usepackage{lipsum} % For generating dummy text
    \usepackage{hyperref} % For hyperlinks
% \usepackage{listings} % For including code snippets
% \usepackage{color} % For defining colors

% Title
\title{Project Proposal -- Stock Prediction}
\author{Uroš Kozole}

% Document start
\begin{document}

\maketitle

% % Abstract (optional)
% \begin{abstract}
% Your abstract text here.
% \end{abstract}

% % Table of contents (optional)
% \tableofcontents

% Sections
\section{Brief description}

Sequential nature of stock prices through time and immense and rising effect of social and other media platforms' on the fluctuations
of the stock market call for approaches that can deal with sequential data and natural language. The field of deep 
learning emerges as the obvious candidate. My goal is to apply one of the approaches to the Slovene stock market.

Based on the survey paper listed in the references I'll determine which approach is most suitable to use for Slovene
stock market. Criteria for how suitable a given approach is will mostly stem from what data is available for Slovene
stock market, and in what capacity (for instance, if a method relied heavily on X trends and I find that not many people
talk about the Slovene stock market on X, such method might not be well suited here).


% Bibliography
\begin{thebibliography}{9}

\bibitem{survey_paper} 
Jinan Zou, Qingying Zhao. 
\textit{Stock Market Prediction via Deep Learning Techniques: A Survey}.
\\\url{https://arxiv.org/abs/2212.12717}
\end{thebibliography}


% End of document
\end{document}
